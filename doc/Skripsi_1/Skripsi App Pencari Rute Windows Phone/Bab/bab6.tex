\chapter{Kesimpulan dan Saran}
\label{chap:Kesimpulan dan Saran}
Bab ini berisi kesimpulan dari pembangunan aplikasi beserta saran untuk pengembangan selanjutnya.

%Kesimpulan
\section{Kesimpulan}
\label{lab:Kesimpulan}
\hspace{0.5cm} Dari hasil penelitian penulis terhadap perancangan aplikasi pencari rute kendaraan umum didapat kesimpulan sebagai berikut.
\begin{enumerate}
	\item Pengguna dapat memasukan lokasi berdasarkan peta, lokasi perangkat, dan dengan kata kunci.
	\item Penggunaan Kiri API sudah dapat digunakan untuk mencari rute angkutan umum.
	\item Akses internet mempengaruhi performansi mendapatkan rute.
\end{enumerate}

%Saran
\section{Saran}
\label{lab:Saran}
\hspace{0.5cm} Berdasarkan hasil kesimpulan yang telah dipaparkan, penulis memberi saran sebagai berikut.
\begin{enumerate}
	\item Memanfaatkan tombol "enter" untuk memanggil \textit{method startRoute}. Masukan dari pengguna karena setelah pengguna menentukan tujuan tombol "Find" terhalang \textit{keyboard} maka pengguna harus menekan tombol "back" terlebih dahulu untuk menekan tombol "Find".
	\item Memanfaatkan pencarian lokasi pada peta agar pengguna dapat memilih lokasi dengan lebih mudah.
	\item Menambahkan \textit{gesture} dan animasi yang akan meningkatkan interkasi pengguna terhadap aplikasi.
	\item Mengubah beberapa tampilan yang menjadi ciri khas Windows Phone seperti menggunakan \textit{application bar}.
\end{enumerate}