\chapter{Kesimpulan dan Saran}
\label{chap:Kesimpulan dan Saran}
Bab ini berisi kesimpulan dari pembangunan aplikasi beserta saran untuk pengembangan selanjutnya.

%Kesimpulan
\section{Kesimpulan}
\label{lab:Kesimpulan}
\hspace{0.5cm} Dari hasil penelitian penulis terhadap perancangan aplikasi pencari rute kendaraan umum didapat kesimpulan sebagai berikut.
\begin{enumerate}
	\item Pengguna dapat memasukan lokasi berdasarkan peta, lokasi perangkat, dan dengan kata kunci.
	\item Penggunaan Kiri API sudah dapat digunakan untuk mencari rute angkutan umum.
	\item Kecepatan internet dan GPS mempengaruhi performansi mendapatkan rute. Kecepatan akses internet yang cepat dan keadaan perangkat yang berada di luar ruangan menyebabkan pencarian rute lebih cepat dilakukan.
	\item Hasil pengujian menunjukan orang-orang lebih memilih untuk mengetikan kata kunci dari pada memilih lokasi pada peta.
\end{enumerate}

%Saran
\section{Saran}
\label{lab:Saran}
\hspace{0.5cm} Berdasarkan hasil kesimpulan yang telah dipaparkan, untuk pengembangan selanjutnya disaran untuk:
\begin{enumerate}
	\item Memanfaatkan tombol "enter" untuk memanggil \textit{method} startRoute. Masukan dari pengguna karena setelah pengguna menentukan tujuan tombol "Find" terhalang \textit{keyboard} maka pengguna harus menekan tombol "back" terlebih dahulu untuk menekan tombol "Find".
	\item Memanfaatkan pencarian lokasi pada peta agar pengguna dapat memilih lokasi dengan lebih mudah.
	\item Menambahkan \textit{gesture} dan animasi yang akan meningkatkan interaksi pengguna terhadap aplikasi.
	\item Mengubah beberapa tampilan yang menjadi ciri khas Windows Phone seperti menggunakan \textit{application bar}.
\end{enumerate}