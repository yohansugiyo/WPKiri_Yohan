\chapter{Pendahuluan}
\label{chap:intro}

Pada bab satu dibahas pendahuluan dari penelitian yang dilakukan. Bab satu terbagi dalam enam sub-bab, yaitu \textit{latar belakang}, \textit{rumusan masalah}, \textit{tujuan}, \textit{batasan masalah}, \textit{ruang lingkup masalah}, \textit{metode penelitian}, \textit{teknik pengumpulan data}, dan \textit{sistematika penulisan}.

\section{Latar Belakang}
\label{sec:latar_belakang}
%Pragraf 1 latar belakang berisi alasan & cerita
\hspace{0.5cm} Transportasi menjadi bagian yang penting bagi kehidupan manusia. Ada dua jenis transportasi yaitu kendaraan umum dan kendaraan pribadi. Pada saat penelitian ini dilakukan, kebanyakan orang lebih memilih kendaraan pribadi dibanding kendaraan umum. Maraknya penggunaan kendaraan pribadi ditambah penambahan jalur kendaraan yang tidak sebanding dengan banyaknya kendaraan di Indonesia akhirnya menimbulkan kemacetan. Hal yang menimbulkan maraknya penggunaan kendaraan pribadi dikarenakan kurang nyamannya kendaraan umum dan kesulitan dalam menentukan kendaraan umum yang harus dinaiki. Kesulitan seseorang untuk memilih kendaraan umum disebabkan banyaknya rute kendaraan umum membuat orang kebingungan dalam memilih kendaraan umum untuk menuju lokasi yang diinginkan dan kurangnya publikasi mengenai rute angkutan umum. Orang tidak tahu mengenai jenis angkutan umum yang akan dipakai dan cenderung malas untuk bertanya dan mencari rute yang benar dan efisien. Karena hal tersebut, seseorang lebih memilih menggunakan kendaraan pribadi dibandingkan dengan kendaraan umum. 

%Pragraf 2 latar belakang berisi keinginan pemilihan topik berdasarkan Kiri API
Ide pembuatan aplikasi yang memudahkan seseorang dalam menentukan rute kendaraan umum sudah lebih dulu ada. Ide dalam penentuan rute angkutan umum dicetuskan oleh Kiri. Pembuat Kiri membuat pencarian rute berdasarkan latar belakang tiga masalah besar yaitu pemanasan global, kemacetan, dan harga bahan bakar minyak yang tinggi\footnotemark[1]. Kiri pertama dikenal dalam bentuk situs \textit{web} dengan alamat \url{kiri.travel}. Meskipun Kiri dibuat dalam basis situs \textit{web}, tetapi layanan Kiri dalam menentukan rute kendaraan umum dapat dimanfaatkan untuk pencarian rute selain dari situs \textit{web}. Pemanfaatan layanan Kiri dalam mencari rute kendaraan umum tersebut yaitu dengan menggunakan Kiri API. Kiri API merupakan serangkaian aturan yang dapat dimanfaatkan dari perangkat lain yang dalam kasus ini adalah untuk mencari rute.
\footnotetext[1]{\url{http://static.kiri.travel/en-about/}} 

%Pragraf 3 latar belakang berisi keinginan pemilihan topik berdasarkan Windows Phone
Pesatnya perkembangan teknologi informasi dan komunikasi memiliki pengaruh besar terhadap kehidupan manusia. Kebutuhan akan informasi mendorong perkembangan perangkat yang dapat mengakses informasi di manapun dan kapanpun. Teknologi yang dapat membantu manusia dalam hal tersebut adalah perangkat \textit{mobile}. Kemudahan dalam penggunaan yang dimiliki perangkat \textit{mobile} kian digemari orang-orang terutama di Indonesia. Beberapa contoh sistem operasi pada perangkat \textit{mobile} adalah Android, IOS, dan Windows Phone. Pada sistem operasi Android sudah banyak aplikasi pencarian rute. Selain Android salah satu sistem operasi yang menarik perhatian adalah Windows Phone 8 yang dibuat Microsoft. Antarmuka Windows Phone 8 Microsoft sebut \textit{Metro}. Antarmuka \textit{Metro} memiliki keunggulan, yaitu menarik dan mudah digunakan. Pada sistem operasi Windows Phone belum dijumpai aplikasi pencari rute.

%Pragraf 4 latar belakang berisi ketertarikan penulis
Berdasarkan hal tersebut, aplikasi yang dikembangkan dalam tugas akhir ini adalah aplikasi Pencarian Rute Kendaraan Umum di Windows Phone. Aplikasi yang dikembangkan memungkinkan pengguna menemukan rute kendaraan umum untuk sampai di tujuan. Untuk memudahkan pengguna, aplikasi akan menampilkan hasil pencarian rute dalam dua bentuk yaitu bentuk peta dan bentuk daftar. 

\section{Rumusan Masalah}
\label{sec:rumusan_masalah}
Sehubung dengan latar belakang pada sub-bab ~\ref{sec:latar_belakang} timbul permasalahan sebagai berikut:
\begin{itemize}
	\item Bagaimana membuat aplikasi di Windows Phone?
	\item Bagaimana mengintegrasikan Kiri API dengan aplikasi pencari rute kendaraan umum di Windows Phone?
\end{itemize}

\section{Tujuan}
\label{sec:tujuan}
Berdasarkan rumusan masalah pada sub-bab 1.2, maka tujuan dari pembuatan tugas akhir ini adalah:
\begin{itemize}
	\item Mempelajari cara pembuatan perangkat lunak di Windows Phone lalu mengembangkan aplikasi yang dibuat.
	\item Membuat aplikasi di Windows Phone yang memanfaatkan Kiri API.
\end{itemize}

\section{Batasan Masalah}
\label{sec:batasan_masalah}
Ruang lingkup pengembangan aplikasi Pencari Rute Kendaraan untuk Windows Phone ini dibatasi hal berikut:
\begin{itemize}
	\item Aplikasi ini membutuhkan koneksi internet.
	\item Aplikasi ini menampilkan rute jalur angkot, bus umum dan travel di empat kota besar yaitu Bandung, Jakarta, Malang, dan Surabaya.  
\end{itemize}

\section{Metode Penelitian}
\label{sec:metode_penelitian}
Metode penelitian yang digunakan dalam membuat tugas akhir ini adalah sebagai berikut:
\begin{itemize}
	\item Melakukan studi pustaka mengenai XAML, kontrol dan navigasi di Windows Phone, Peta di Windows Phone, GPS di Windows Phone dan Kiri API.
	\item Melakukan analisis terhadap aplikasi lain yang menggunakan Kiri API.
	\item Melakukan analisis terhadap dasar teori untuk pembangunan aplikasi Pencarian Rute Kendaraan Umum untuk Windows Phone.
	\item Melakukan perancangan aplikasi Pencarian Rute Kendaraan Umum untuk Windows Phone.
	\item Implementasi dari aplikasi Pencarian Rute Kendaraan Umum untuk Windows Phone.
	\item Menguji aplikasi Pencarian Rute Kendaraan Umum untuk Windows Phone.
	\item Membuat kesimpulan.
\end{itemize}

%\section{Teknik Pengumpulan Data}
%\label{sec:teknik_pengumpulan_data}

\section{Sistematika Penulisan}
\label{sec:sistematika_penulisan}
%Bab 1
\hspace{0.5cm} Bab 1 membahas latar belakang masalah, rumusan masalah, tujuan penulisan tugas akhir, batasan masalah, ruang lingkup masalah, metode penelitian, dan teknik pengumpulan data tugas akhir ini.

%Bab 2
Bab 2 membahas tentang teori-teori yang digunakan dalam tugas akhir ini. Bahasan yang dijelaskan pada bab ini adalah Windows Phone dan Kiri API. Teori Windows Phone yang dijelaskan meliputi lingkungan kerja, XAML, kontrol terhadap ponsel, siklus hidup aplikasi, peta di Windows Phone, lokasi, dan memanfaatkan sumber data. Teori Kiri API yang dijelaskan meliputi \textit{web service} penentuan rute, \textit{web service} pencarian lokasi, dan \textit{web service} menemukan transportasi terdekat. 

%Bab 3
Bab 3 membahas tentang analisis pembangunan aplikasi Pencarian Rute Kendaraan Umum untuk Windows Phone. Pada Bab 3 dibahas mengenai analisis kebutuhan aplikasi, analisis kontrol yang dipakai, analisis terhadap siklus hidup aplikasi, analisis terhadap siklus hidup aplikasi, analisis peta, analisis memanfaatkan sumber data, analisis Kiri API, diagram \textit{use case}, dan diagram kelas.

%Bab 4
Bab 4 membahas tentang perancangan aplikasi dari pembahasan di bab 2 dan bab 3. Bab perancangan dimulai dengan perancangan diagram \textit{sequence} yang dapat menggambarkan interaksi antar objek, diagram kelas digunakan untuk menggambarkan hubungan antar suatu kelas, dan antarmuka aplikasi.

%Bab 5
Bab 5 membahas tentang implementasi  dari perancangan di bab 4 lalu pengujian yang sudah dilakukan. Pengujian yang dilakukan ada dua yaitu pengujian fungsional dan eksprimental. 

%Bab 6
Bab 6 membahas tentang kesimpulan dari penelitian ini dan saran dari penulis terhadap penelitian ini.