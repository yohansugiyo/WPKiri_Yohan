\chapter{Implementasi dan Pengujian Aplikasi}
\label{chap:Implementasi dan Pengujian Aplikasi}

Pada bab 5 akan dibahas implementasi dan pengujian aplikasi pencari rute kendaraan umum untuk Windows Phone.

%Implementasi
\section{Implementasi}
\label{lab:Implementasi}
\hspace{0.5cm} Pada sub bab ini akan dijelaskan mengenai ligkungan yang digunakan untuk membangun aplikasi Pencari Rute Kendaraan Umum untuk Windows Phone. Pada lingkungan yang akan dibahas juga penulis membangun aplikasi sesuai rancangan yang telah dibahas pada bab 4 dan mengujinya.

%Perangkat Keras untuk Implementasi
\subsection{Perangkat Keras untuk Implementasi}
\label{lab:Perangkat Keras untuk Implementasi}
\hspace{0.5cm} Dalam membangun aplikasi ini perangkat keras yang digunakan adalah sebagai berikut:
\begin{enumerate}
	\item Komputer
		\begin{enumerate}
			\item Processor: intel Core i7-2620M CPU 2,7 GHz
			\item RAM: 4 GB
			\item Hardisk: 640 GB
			\item VGA: Intel HD 3000
		\end{enumerate}
		
	\item Perangkat Bergerak
		\begin{enumerate}
			\item Processor: 1,2 GHz
			\item RAM: 1 GB
			\item ROM: 8 GB
			\item Layar: 720 x 1280 pixel, 4,7 inch
			\item GPS
			\item Sensor: kompas, \textit{accelerometer}
		\end{enumerate}
\end{enumerate}

%Perangkat Lunak untuk Implementasi
\subsection{Perangkat Lunak untuk Implementasi}
\label{lab:Perangkat Lunak untuk Implementasi}
\hspace{0.5cm} Dalam membangun aplikasi ini perangkat lunak yang digunakan adalah sebagai berikut:
\begin{enumerate}
	\item Komputer
		\begin{enumerate}
			\item Sistem Operasi
			\item IDE
			\item Bahasa Pemrograman
			\item Library
		\end{enumerate}
		
	\item Perangkat Bergerak
		\begin{enumerate}
			\item Sistem Operasi
		\end{enumerate}
\end{enumerate}

%Hasil Implementasi
\subsection{Hasil Implementasi}
\label{lab:Hasil Implementasi}

%Pengujian
\section{Pengujian}
\label{lab:Pengujian}
\hspace{0.5cm} 

%Lingkungan Pengujian
\subsection{Lingkungan Pengujian}
\label{lab:Lingkungan Pengujian}

%Pengujian Fungsional untuk mengetahui kesesuaian reaksi nyata dengan yang diharapkan
\subsection{Pengujian Fungsional}
\label{lab:Pengujian Fungsional}

%Pengujian Experimental untuk mengetahui tingkat keberhasilan proses kerja aplikasi
\subsection{Pengujian Experimental}
\label{lab:Pengujian Experimental}