\chapter{Perancangan}
\label{chap:Perancangan}

Pada bab 4 akan dibahas mengenai perancangan seperti kelas secara rinci dan perancangan antarmuka.

%Perancangan Kelas
\section{Perancangan Kelas}
\label{lab:Perancangan Kelas}
\hspace{0.5cm} Pada sub bab ini akan dibahas mengenai deskripsi kelas secara rinci pada aplikasi Pencari Rute Kendaraan Umum untuk Windows Phone. Untuk lebih jelas mengenai kelas yang ada pada aplikasi ini, penulis menyajikan gambar diagram kelas yang dapat dilihat pada  gambar ~\ref{fig:kelas}. 

% Kelas
\begin{figure}[h]
	\centering
		\includegraphics[scale=0.2]{Gambar/useCase_dan_Class/class4}
	\caption{Diagram Kelas}
	\label{fig:kelas}
\end{figure}

%Kelas PhoneApplicationPage
\subsection{Kelas \textit{PhoneApplicationPage}}
\label{lab:Kelas PhoneApplicationPage}
\hspace{0.5cm} \textit{PhoneApplicationPage} merupakan kelas bawaan Windows Phone yang menangani interksi pengguna dengan aplikasi dan siklus hidup aplikasi.

%Kelas MainPage
\subsection{Kelas \textit{MainPage}}
\label{lab:Kelas MainPage}
\hspace{0.5cm} \textit{MainPage} merupakan kelas kelas turunan dari kelas \textit{PhoneApplicationPage} yang menangani interaksi langsung antara halaman aplikasi dengan pengguna. Pada kelas ini akan ditaruh kontrol yang diperlukan. Berikut adalah penjelasan atribut-atribut yang dimiliki kelas ini:
\begin{enumerate}
	\item protocol bertipe Protocol untuk mendapatkan URL yang digunakan dalam permintaan ke Kiri API.
	\item lFinder bertipe LocationFinder yang akan menampung objek untuk pencarian lokasi.
	\item httpClient bertipe HttpClient merupakan objek yang akan mengurus permintaan dan kembalian dari Kiri API.
	\item listPlaceFrom bertipe ListBox untuk menampung dan menampilkan hasil pencarian lokasi terkait dari lokasi awal.
	\item listPlaceTo bertipe ListBox untuk menampung dan menampilkan hasil pencarian lokasi terkait dari lokasi tujuan.
	\item locFrom bertipe string untuk menampung kordinat awal.
	\item locTo bertipe string untuk menampung kordinat tujuan.
	\item centerOfBandung bertipe Point untuk menampung pusat kordinat kota Bandung.
	\item centerOfJakarta bertipe Point untuk menampung pusat kordinat kota Jakarta.
	\item centerOfMalang bertipe Point untuk menampung pusat kordinat kota Malang.
	\item centerOfSurabaya bertipe Point untuk menampung pusat kordinat kota Surabaya.
	\item city bertipe kumpulan String untuk menampung kota yang didukung oleh layanan Kiri.
	\item myCity bertipe String untuk menampung kode kota sesuai Kode Penerbangan IATA.
\end{enumerate}

Berikut adalah penjelasan metode-metode yang dimiliki kelas ini:
\begin{enumerate}
	\item Metode MainPage digunakan sebagai konstruktor pada kelas MainPage. 
	\item Metode startRoute digunakan untuk mendapatkan masukan pengguna lalu mengkalkulasi masukan tersebut menjadi kordinat asal dan tujuan lalu mengirimkannya ke kelas ShowRoute.
	\item Metode toObjectSearchPlace digunakan untuk membuat objek RootObjectSearchPlace dengan masukan uri yang bertipe string.
	\item Metode changeMapFrom digunakan untuk berpindah ke halaman mapFrom.
	\item Metode changeMapTo digunakan untuk berpindah ke halaman mapTo.
	\item Metode getHere digunakan untuk mendapatkan kordinat dimana perangkat berada.
	\item Metode getListItem digunakan untuk membuat listBox lalu menampilkan ke pengguna. 
	\item Metode ListBoxSelectedPlaceFrom digunakan untuk mendapatkan tempat asal yang dipilih pengguna.
	\item Metode ListBoxSelectedPlaceTo digunakan untuk mendapatkan tempat tujuan yang dipilih pengguna.
	\item Metode searchCoordinatePlace digunakan untuk mencari kordinat dari tempat pilihan pengguna.
	\item Metode changeCity digunakan untuk mengubah kota tujuan dari pencarian.
\end{enumerate}

%Kelas Geocoordinate
\subsection{Kelas \textit{Geocoordinate}}
\label{lab:Kelas Geocoordinate}
\hspace{0.5cm} \textit{Geocoordinate} merupakan kelas bawaan dari Windows Phone yang akan dimanfaatkan untuk membaca \textit{latitude} dan \textit{altitude}.

%Kelas Geolocator
\subsection{Kelas \textit{Geolocator}}
\label{lab:Kelas Geolocator}
\hspace{0.5cm} \textit{Geolocator} merupakan kelas bawaan Windows Phone untuk mengkases lokasi. Dengan bantuan kelas ini maka dapat mengetahui status lokasi dari perangkat dan menemukan lokasi secara akurat.

%Kelas Geoposition
\subsection{Kelas \textit{Geoposition}}
\label{lab:Kelas Geoposition}
\hspace{0.5cm} \textit{Geoposition} merupakan kelas yang menampung lokasi sesuak kembalian \textit{Geolocator}.

%Kelas LocationFinder
\subsection{Kelas \textit{LocationFinder}}
\label{lab:Kelas LocationFinder}
\hspace{0.5cm} \textit{LocationFinder} merupakan kelas yang akan menampung lokasi perangkat. Berikut adalah penjelasan atribut-atribut yang dimiliki kelas ini:
\begin{enumerate}
	\item coorLat bertipe Double untuk menampung kordinat latitude.
	\item coorLong bertipe Double untuk menampung kordinat longitude.
	\item myGeoCoordinate bertipe GeoCoordinate untuk menampung lokasi perangkat
\end{enumerate}

Berikut adalah penjelasan metode-metode yang dimiliki kelas ini:
\begin{enumerate}
	\item Metode OneShotLocation\_Click berfungsi inisialisasi GPS lalu mendapat kordinat dan menampungnya di atribut.
\end{enumerate}

%Kelas PointFromMap
\subsection{Kelas \textit{PointFromMap}}
\label{lab:Kelas PointFromMap}
\hspace{0.5cm} \textit{PointFromMap} merupakan kelas yang akan mendapatkan titik yang ditunjuk pengguna pada peta lalu menerjemahkannya dalam bentuk titik kordinat. Berikut adalah penjelasan atribut-atribut yang dimiliki kelas ini:
\begin{enumerate}
	\item latitude
	\item longitude
\end{enumerate}

Berikut adalah penjelasan metode-metode yang dimiliki kelas ini:
\begin{enumerate}
	\item Metode getPoint berfungsi mengambil titik yang ditunjuk lalu menerjemahkan dalam bentuk latitude dan longitude. 
\end{enumerate}

%Kelas HttpClient
\subsection{Kelas \textit{HttpClient}}
\label{lab:Kelas HttpClient}
\hspace{0.5cm} \textit{HttpClient} merupakan kelas bawaan Windows Phone untuk mengatur pengiriman dan kembalian menggunakan protokol HTTP. Berikut adalah penjelasan metode-metode kelas \textit{HttpClient} yang dipakai untuk perancangan aplikasi ini:
\begin{enumerate}
	\item Metode GetStringAsync membutuhkan parameter alamat bertipe string dan mengembalikan kembalian dari Kiri dalam bentuk Task<string>.
\end{enumerate}

%Kelas Protocol
\subsection{Kelas \textit{Protocol}}
\label{lab:Kelas Protocol}
\hspace{0.5cm} \textit{Protocol} merupakan kelas untuk menampung semua alamat dalam pengiriman menggunakan protokol HTTP. Berikut adalah penjelasan atribut-atribut yang dimiliki kelas ini:
\begin{enumerate}
	\item apiKey bertipe string digunakan untuk menyimpan kunci untuk mengirim permintaan ke Kiri.
	\item hostname bertipe string digunakan untuk digunakan untuk menyimpan alamat host dari Kiri.
	\item handle bertipe string digunakan untuk menyimpan alamat host ditambah "handle.php".
	\item iconPath bertipe string digunakan untuk menyimpan lokasi gambar yang dibutuhkan.
	\item iconStart bertipe string digunakan untuk menyimpan lokasi gambar awal perjalanan dari lokasi awal.
	\item iconFinish bertipe string digunakan untuk menyimpan lokasi gambar akhir perjalanan ke lokasi tujuan.
	
	\item version bertipe string digunakan untuk menyimpan veris dari API yang digunakan.
	
	\item modeFind bertipe string yang digunakan untuk menyimpan mode mencari lokasi terkait pada Kiri API.
	\item modeRoute bertipe string yang digunakan untuk menyimpan mode mencari rute pada Kiri API.
	\item modeNearby bertipe string yang digunakan untuk menyimpan mode mencari lokasi terdekat pada Kiri API.
	
	\item locale bertipe string yang digunakan untuk menyimpan kata "locale".
	
	\item start bertipe string yang digunakan untuk menyimpan kata "start".
	\item finish bertipe string yang digunakan untuk menyimpan kata "finish".
	
	\item presentation bertipe string yang digunakan untuk menyimpan kata "presentation".
	
	\item region bertipe string yang digunakan untuk menyimpan kata "region".
\end{enumerate}

Berikut adalah penjelasan metode-metode yang dimiliki kelas ini:
\begin{enumerate}
	\item getTypeTransport merupakan metode yang akan mengembalikan alamat dari gambar transportasi. Metode ini memiliki 2 parmeter yaitu means sebagai tipe transportasi dan meansDetail sebagai nama kendaraan.
	\item getSearchPlace merupakan metode yang akan mengembalikan URI pencarian lokasi sesuai paramater. Parameter yang dimaksud adalah \textit{query} masukan pengguna.
	\item getFindRoute merupakan metode yang akan mengembalikan URI pencarian rute sesuai parameter. Parameter yang dimaksud adalah kordinat lokasi asal dan kordinat lokasi tujuan yang bertipe string.
\end{enumerate}

%Kelas RootObjectSearchPlace
\subsection{Kelas \textit{RootObjectSearchPlace}}
\label{lab:Kelas RootObjectSearchPlace}
\hspace{0.5cm} \textit{RootObjectSearchPlace} merupakan kelas untuk menampung objek hasil pencarian lokasi. Berikut adalah penjelasan atribut-atribut yang dimiliki kelas ini:
\begin{enumerate}
	\item status bertipe \textit{string} digunakan untuk menampung hasil kembalian status dari Kiri.
	\item searchresult bertipe \textit{list} dan menampung banyak objek SearchResult. 
	\item attributions bertipe objek untuk menampung attributions.
\end{enumerate}


%Kelas SearchResult
\subsection{Kelas \textit{SearchResult}}
\label{lab:Kelas SearchResult}
\hspace{0.5cm} \textit{SearchResult} merupakan kelas untuk menampung nama tempat dan kordinat dari nama tempat tersebut. Berikut adalah penjelasan atribut-atribut yang dimiliki kelas ini:
\begin{enumerate}
	\item placename bertipe \textit{string} digunakan untuk menampung nama tempat. 
	\item location bertipe \textit{string} digunakan untuk menampung nama tempat.
\end{enumerate}

%Kelas RootObjectFindRoute
\subsection{Kelas \textit{RootObjectFindRoute}}
\label{lab:Kelas RootObjectFindRoute}
\hspace{0.5cm} \textit{RootObjectFindRoute} merupakan kelas untuk menampung hasil pencarian rute. Berikut adalah penjelasan atribut-atribut yang dimiliki kelas ini:
\begin{enumerate}
	\item status
	\item routingresults
\end{enumerate}

%Kelas RoutingResult
\subsection{Kelas \textit{RoutingResult}}
\label{lab:Kelas RoutingResult}
\hspace{0.5cm} \textit{RoutingResult} merupakan kelas untuk menampung langkah menuju tempat tujuan dan waktu yang dibutuhkan. Berikut adalah penjelasan atribut-atribut yang dimiliki kelas ini:
\begin{enumerate}
	\item steps
	\item traveltime
\end{enumerate}


%Perancangan Antar Muka
\section{Perancangan Antar Muka}
\label{lab:Perancangan Kelas}
\hspace{0.5cm} Pada sub bab ini akan dibahas mengenai antarmuka pada aplikasi Pencari Rute Kendaraan Umum untuk Windows Phone. Antarmuka berfungsi sebagai jembatan yang menghubungkan antara aplikasi dengan pengguna. Berikut ini akan dijelaskan mengenai rancangan antarmuka aplikasi Pencari Rute Kendaraan Umum untuk Windows Phone. 

%Kelas MainPage
\subsection{Antarmuka Kelas \textit{MainPage}}
\label{lab:Antarmuka Kelas MainPage}

% Antarmuka Main
\begin{figure}[h]
	\centering
		\includegraphics[scale=0.2]{Gambar/antarmuka/home}
	\caption{Antarmukan \textit{MainPage}}
	\label{fig:Antarmuka MainPage}
\end{figure}

\hspace{0.5cm} \textit{Antarmuka Kelas MainPage} 

%Kelas MapFrom
\subsection{Antarmuka Kelas \textit{MapFrom}}
\label{lab:Antarmuka Kelas MapFrom}

% Antarmuka Main
\begin{figure}[h]
	\centering
		\includegraphics[scale=0.2]{Gambar/antarmuka/map}
	\caption{Antarmukan \textit{MainPage}}
	\label{fig:Antarmuka MainPage}
\end{figure}

\hspace{0.5cm} \textit{Antarmuka Kelas MapFrom} 

%Kelas MapTo
\subsection{Antarmuka Kelas \textit{MapTo}}
\label{lab:Antarmuka Kelas MapFrom}

% Antarmuka MapTo
\begin{figure}[h]
	\centering
		\includegraphics[scale=0.2]{Gambar/antarmuka/map}
	\caption{Antarmukan \textit{MapTo}}
	\label{fig:Antarmuka MapTo}
\end{figure}

\hspace{0.5cm} \textit{Antarmuka Kelas MapTo} 

%Kelas FindRoute
\subsection{Antarmuka Kelas \textit{FindRoute}}
\label{lab:Antarmuka Kelas FindRoute}

% Antarmuka MapTo
%\begin{figure}[h]
%	\centering
%		\includegraphics[scale=0.2]{Gambar/antarmuka/map}
%	\caption{Antarmukan \textit{FindRoute}}
%	\label{fig:Antarmuka FindRoute}
%\end{figure}

\hspace{0.5cm} \textit{Antarmuka Kelas FindRoute} 