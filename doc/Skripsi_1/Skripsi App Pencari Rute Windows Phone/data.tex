%_____________________________________________________________________________
%=============================================================================
% data.tex v6 (13-04-2015) \ldots dibuat oleh Lionov - Informatika FTIS UNPAR
%
% Perubahan pada versi 6 (13-04-2015)
% - Perubahan untuk data-data ``template" menjadi lebih generik dan menggunakan
%	tanda << dan >>
%
% Perubahan pada versi sebelumnya
% 	versi 5 (10-11-2013)
% 	- Perbaikan pada memasukkan bab : tidak perlu menuliskan apapun untuk 
%	  memasukkan seluruh bab (bagian V)
% 	- Perbaikan pada memasukkan lampiran : tidak perlu menuliskan apapun untuk
%	  memasukkan seluruh lampiran atau -1 jika tidak memasukkan apapun
%	versi 4 (21-10-2012)
%	- Data dosen dipindah ke dosen.tex agar jika ada perubahan/update data dosen
%   mahasiswa tidak perlu mengubah data.tex
%	- Perubahan pada keterangan dosen	
% 	versi 3 (06-08-2012)
% 	- Perubahan pada beberapa keterangan 
% 	versi 2 (09-07-2012):
% 	- Menambahkan data judul dalam bahasa inggris
% 	- Membuat bagian khusus untuk judul (bagian VIII)
% 	- Perbaikan pada gelar dosen
%_____________________________________________________________________________
%=============================================================================
% 								BAGIAN -
%=============================================================================
% Ini adalah file data (data.tex)
% Masukkan ke dalam file ini, data-data yang diperlukan oleh template ini
% Cara memasukkan data dijelaskan di setiap bagian
% Data yang WAJIB dan HARUS diisi dengan baik dan benar adalah SELURUHNYA !!
% Hilangkan tanda << dan >> jika anda menemukannya
%=============================================================================
%_____________________________________________________________________________
%=============================================================================
% 								BAGIAN I
%=============================================================================
% Tambahkan package2 lain yang anda butuhkan di sini
%=============================================================================
\usepackage{booktabs}
\usepackage[table]{xcolor}
\usepackage{longtable}
\usepackage{amsmath}
%=============================================================================

%_____________________________________________________________________________
%=============================================================================
% 								BAGIAN II
%=============================================================================
% Mode dokumen: menetukan halaman depan dari dokumen, apakah harus mengandung 
% prakata/pernyataan/abstrak dll (termasuk daftar gambar/tabel/isi) ?
% - kosong : tidak ada halaman depan sama sekali (untuk dokumen yang 
%            dipergunakan pada proses bimbingan)
% - cover : cover saja tanpa daftar isi, gambar dan tabel
% - sidang : cover, daftar isi, gambar, tabel (IT: UTS-UAS Seminar 
%			 dan UTS TA)
% - sidang_akhir : mode sidang + abstrak + abstract
% - final : seluruh halaman awal dokumen (untuk cetak final)
% Jika tidak ingin mencetak daftar tabel/gambar (misalkan karena tidak ada 
% isinya), edit manual di baris 439 dan 440 pada file main.tex
%=============================================================================
% \mode{kosong}
% \mode{cover}
% \mode{sidang}
%\mode{sidang_akhir}
\mode{final} 
%=============================================================================

%_____________________________________________________________________________
%=============================================================================
% 								BAGIAN III
%=============================================================================
% Line numbering: penomoran setiap baris, otomatis di-reset setiap berganti
% halaman
% - yes: setiap baris diberi nomor
% - no : baris tidak diberi nomor, otomatis untuk mode final
%=============================================================================
\linenumber{yes}
%=============================================================================

%_____________________________________________________________________________
%=============================================================================
% 								BAGIAN IV
%=============================================================================
% Linespacing: jarak antara baris 
% - single: opsi yang disediakan untuk bimbingan, jika pembimbing tidak
%            keberatan (untuk menghemat kertas)
% - onehalf: default dan wajib (dan otomatis) jika ingin mencetak dokumen
%            final/untuk sidang.
% - double : jarak yang lebih lebar lagi, jika pembimbing berniat memberi 
%            catatan yg banyak di antara baris (dianjurkan untuk bimbingan)
%=============================================================================
%\linespacing{single}
 \linespacing{onehalf}
%\linespacing{double}
%=============================================================================

%_____________________________________________________________________________
%=============================================================================
% 								BAGIAN V
%=============================================================================
% Bab yang akan dicetak: isi dengan angka 1,2,3 s.d 9, sehingga bisa digunakan
% untuk mencetak hanya 1 atau beberapa bab saja
% Jika lebih dari 1 bab, pisahkan dengan ',', bab akan dicetak terurut sesuai 
% urutan bab.
% Untuk mencetak seluruh bab, kosongkan parameter (i.e. \bab{ })  
% Catatan: Jika ingin menambahkan bab ke-10 dan seterusnya, harus dilakukan 
% secara manual
%=============================================================================
\bab{ }
%=============================================================================

%_____________________________________________________________________________
%=============================================================================
% 								BAGIAN VI
%=============================================================================
% Lampiran yang akan dicetak: isi dengan huruf A,B,C s.d I, sehingga bisa 
% digunakan untuk mencetak hanya 1 atau beberapa lampiran saja
% Jika lebih dari 1 lampiran, pisahkan dengan ',', lampiran akan dicetak 
% terurut sesuai urutan lampiran
% Jika tidak ingin mencetak lampiran apapun, isi dengan -1 (i.e. \lampiran{-1})
% Untuk mencetak seluruh mapiran, kosongkan parameter (i.e. \lampiran{ })  
% Catatan: Jika ingin menambahkan lampiran ke-J dan seterusnya, harus 
% dilakukan secara manual
%=============================================================================
\lampiran{ }
%=============================================================================

%_____________________________________________________________________________
%=============================================================================
% 								BAGIAN VII
%=============================================================================
% Data diri dan skripsi/tugas akhir
% - namanpm: Nama dan NPM anda, penggunaan huruf besar untuk nama harus benar
%			 dan gunakan 10 digit npm UNPAR, PASTIKAN BAHWA BENAR !!!
%			 (e.g. \namanpm{Jane Doe}{1992710001}
% - judul : Dalam bahasa Indonesia, perhatikan penggunaan huruf besar, judul
%			tidak menggunakan huruf besar seluruhnya !!! 
% - tanggal : isi dengan {tangga}{bulan}{tahun} dalam angka numerik, jangan 
%			  menuliskan kata (e.g. AGUSTUS) dalam isian bulan
%			  Tanggal ini adalah tanggal dimana anda akan melaksanakan sidang 
%			  ujian akhir skripsi/tugas akhir
% - pembimbing: isi dengan pembimbing anda, lihat daftar dosen di file dosen.tex
%				jika pembimbing hanya 1, kosongkan parameter kedua 
%				(e.g. \pembimbing{\JND}{  } ) , \JND adalah kode dosen
% - penguji : isi dengan para penguji anda, lihat daftar dosen di file dosen.tex
%				(e.g. \penguji{\JHD}{\JCD} ) , \JND dan \JCD adalah kode dosen
%
%=============================================================================
\namanpm{Yohan}{2011730048}	%hilangkan tanda << & >>
\tanggal{28}{5}{2015}			%hilangkan tanda << & >>
\pembimbing{\TAB}{}     
%Lihat singkatan pembimbing anda di file dosen.tex, hilangkan tanda << & >>
\penguji{\JNH}{\LNV} 		
%Lihat singkatan penguji anda di file dosen.tex, hilangkan tanda << & >>
%=============================================================================

%_____________________________________________________________________________
%=============================================================================
% 								BAGIAN VIII
%=============================================================================
% Judul dan title : judul bhs indonesia dan inggris
% - judulINA: judul dalam bahasa indonesia
% - judulENG: title in english
% PERHATIAN: - langsung mulai setelah '{' awal, jangan mulai menulis di baris 
%			   bawahnya
%			 - Gunakan \texorpdfstring{\\}{} untuk pindah ke baris baru
%			 - Judul TIDAK ditulis dengan menggunakan huruf besar seluruhnya !!
%			 - Gunakan perintah \texorpdfstring{\\}{} untuk baris baru
%=============================================================================

\judulINA{Pengembangan Aplikasi Navigasi \texorpdfstring{\\}{} Kendaraan Umum untuk Windows Phone \texorpdfstring{\\}{} Berbasis Kiri API}

\judulENG{Development of Public Transport \texorpdfstring{\\}{} Navigation Application For Windows Phone Based on Kiri API}

%_____________________________________________________________________________
%=============================================================================
% 								BAGIAN IX
%=============================================================================
% Abstrak dan abstract : abstrak bhs indonesia dan inggris
% - abstrakINA: abstrak bahasa indonesia
% - abstrakENG: abstract in english
% PERHATIAN: langsung mulai setelah '{' awal, jangan mulai menulis di baris 
%			 bawahnya
%=============================================================================

\abstrakINA{
	\hspace{0.5cm} Masalah transportasi sering dijumpai di negara-negara di Dunia termasuk di Indonesia. Permasalahan transportasi sebenarnya dapat diatasi dengan penggunaan kendaraan umum. Namun karena sulit dalam mencari rute penggunaan kendaraan umum menjadi sedikit. Penggunaan jumlah kendaraan pribadi yang tidak sebanding ruas jalan menimbulkan kemacetan. Untuk itu orang banyak harus dibantu dalam menentukan rute kendaraan umum, dengan meningkatnya jumlah pengguna angkutan umum diharapkan akan mengurai kemacetan di Indonesia. 
	
	\hspace{0.5cm}Pesatnya perkembangan teknologi membuat perangkat bergerak kian digemari di Indonesia. Salah satu sistem operasi yang meramaikan industri perangkat bergerak adalah Windows Phone. 
	Windows Phone merupakan sistem operasi buatan Microsoft. Salah satu yang membedakan sistem operasi Windows Phone dengan sistem operasi lain adalah antarmuka yang Microsoft sebut Metro.
	Saat penelitian ini dilakukan sistem operasi Windows Phone adalah versi 8. 
	
	\hspace{0.5cm} Aplikasi yang berhasil dibangun pada penelitian ini menggunakan bahasa C\# untuk Windows Phone. Untuk mencari rute angkutan umum penulis memanfaatkan Kiri API untuk mendapatkan rute yang diinginkan pengguna. Aplikasi yang dibuat memanfaatkan protokol HTTP untuk permintaan data lalu mendapatkan kembalian berupa JSON. Antarmuka aplikasi dibuat ringkas dengan masukan lokasi asal dan lokasi tujuan. Masukan dapat berupa  nama jalan atau nama tempat. Setelah lokasi dimasukan maka rute akan diolah oleh Kiri dan aplikasi akan menampilkan hasil pencarian rute dalam 2 bentuk yaitu bentuk daftar dan bentuk peta.
}

\abstrakENG{
	\hspace{0.5cm} Transportation becomes an integral part of human life when this thesis was made. But today many people who prefer to use private transport compared to public transport. Widespread use of private vehicles cause problems of global warming, congestion, and fuel prices are high.
	
	\hspace{0.5cm} The rapid development of technology makes it increasingly popular mobile devices in Indonesia. One of the operating system that enliven the mobile industry is Windows Phone. Windows Phone is Microsoft operating system. One of the differentiating Windows Phone operating system with other operating systems is a Microsoft interface called Metro. Currently this research system Windows Phone operating is version 8.
	
	\hspace{0.5cm} The Application successfully constructed in this study using the language C\# for Windows Phone. To find a public transport route Left writers utilize the web service API and utilize to get the desired route users. The application is made utilizing the HTTP protocol for request data and get a return in the form of JSON. Application interface displays search results in two forms, namely the form of lists and maps.
} 

%=============================================================================

%_____________________________________________________________________________
%=============================================================================
% 								BAGIAN X
%=============================================================================
% Kata-kata kunci dan keywords : diletakkan di bawah abstrak (ina dan eng)
% - kunciINA: kata-kata kunci dalam bahasa indonesia
% - kunciENG: keywords in english
%=============================================================================
\kunciINA{Rute, Kendaraan Umum, Windows Phone}

\kunciENG{Route, Public Transport, Windows Phone}
%=============================================================================

%_____________________________________________________________________________
%=============================================================================
% 								BAGIAN XI
%=============================================================================
% Persembahan : kepada siapa anda mempersembahkan skripsi ini ...
%=============================================================================
\untuk{Dipersembahkan untuk diri sendiri dan orang tua}
%=============================================================================

%_____________________________________________________________________________
%=============================================================================
% 								BAGIAN XII
%=============================================================================
% Kata Pengantar: tempat anda menuliskan kata pengantar dan ucapan terima 
% kasih kepada yang telah membantu anda bla bla bla ....  
%=============================================================================
\prakata{
	Puji Syukur penulis kepada Tuhan yang telah memberikan rahmatnya sehingga 
penulis dapat menyelesaikan skripsi yang berjudul "\textbf{PENGEMBANGAN APLIKASI 
NAVIGASI KENDARAAN UMUM UNTUK WINDOWS PHONE BERBASIS KIRI API}". Skripsi ini disusun 
dengan maksud memenihi salah satu prasyarat menyelesaikan pendidikan di Jurusan Teknik 
Informatika, Fakultas Teknologi Informasi dan Sains, Universitas Katolik Parahyangan.
Penulis menyadari bahwa dalam penulisan skripsi ini tidak terlepas dari bantuan dan dukungan berbagai pihak.
Oleh karena itu, penulis ingin mengucapkan terima kasih kepada:
\begin{itemize}
	\item Pak Thomas Anung Basuki selaku pembimbing yang telah memberikan banyak masukan untuk skripsi ini sehingga skripsi ini dapat diselesaikan dengan baik. 
	\item Pak Pascal Alfadian atas masukan dan tambahan wawasan untuk skripsi ini sehingga skripsi ini dapat diselesaikan dengan baik. 
	\item Orang tua penulis yang selalu memberi dukungan baik moril maupun materil dan doa.
	\item Adik penulis yaitu Yovin yang selalu memberi dukungan dan doa.
	\item Ibu Joanna Helga dan Pak Lionov selaku pengujiyang sudah memberikan banyak masukan dalam penyusunan skripsi ini.
	\item Segenap teman penulis Alexius, Tommy, Winel, Jovan, Kevin Ferdi, Rico, David, Sam, Radit, Neta, Mario, Oswin, Edbert, Kiki, Jeane, Vinny, Reanta, Putri, Vania, Melissa, Willianto
	\item Segenap dosen Jurusan Teknik Informatika Universitas Katolik Parahyangan yang terlah memberikan ilmu untuk penulis.
	\item Seluruh teman jurusan Teknik Informatika, Fakultas Teknologi Informasi dan Sains Universitas Katolik Parahyangan.
\end{itemize}
Semoga segala bantuan dan dukungan berbagai pihak tersebut mendapat berkat dari Tuhan. Semoga skripsi ini berguna bagi semua orang dan dapat dijadikan bahan pembelajaran. Akhir kata, penulis mohon maaf apabila kesalahan dan kekurangan dalam penulisan skripsi ini. 
}
%=============================================================================

%_____________________________________________________________________________
%=============================================================================
% 								BAGIAN XIII
%=============================================================================
% Tambahkan hyphen (pemenggalan kata) yang anda butuhkan di sini 
%=============================================================================
\hyphenation{ma-te-ma-ti-ka}
\hyphenation{fi-si-ka}
\hyphenation{tek-nik}
\hyphenation{in-for-ma-ti-ka}
%=============================================================================


%=============================================================================
